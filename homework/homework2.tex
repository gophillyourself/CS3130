\documentclass[12pt,largemargins]{homework}

  \newcommand{\hwname}{Phillip Janowski}
  \newcommand{\hwemail}{pajmc2@mail.umsl.edu}
  \newcommand{\hwtype}{Homework}
  \newcommand{\hwnum}{2}
  \newcommand{\hwlecture}{E}
  \newcommand{\hwsection}{01}
  \newcommand{\hwclass}{CS3130}

\usepackage{graphicx}
\usepackage{makecell}
\usepackage{enumitem}
\usepackage{multirow}
\usepackage{amsmath}
\usepackage{listings}
\usepackage{forest}


\date{September 6, 2018}
\begin{document}
\maketitle
\question
Explain, if $3^{n + 1}$ is: $ O(3^n ) ; \Omega ( 3^n ) ; \Theta ( 3^n ) $.\\
For $O(3^n)$\\
$ 0\leq 3^{n+1} \leq C*3^n$\\
$3^{n+1} = 3*3^n = C * 3^n \therefore 3^{n+1} \in O(3^n)$\\
For $\Omega (3^n)$\\
$3^{n+1} \geq C*3^n$\\
$3^{n+1} = 3*3^n = C * 3^n \therefore 3^{n+1} \in \Omega(3^n)$\\
NOT DONE $\downarrow$\\
For $\Theta (3^n)$\\
$C_1 * 3^n \leq 3^{n+1} \leq C_2*3^n$\\
$3^{n+1} = 3*3^n = C * 3^n \therefore 3^{n+1} \in \Theta(3^n)$\\

\question
Explain, if $3^{3 n}$ is: $ O ( 3^n ) ; \Omega ( 3^n ) ; \Theta ( 3^n ).$\\
$3^{3n} \notin O(3^n), \in \Omega(3^n), \notin \Theta(3^n)$\\
$3^{3n}$ will always grow faster than any $C * 3^n$ therefore it will only be in $\Omega(3^n)$ $f(n)\geq C*g(n)$\\
\question
\begin{tabular}{|c|c|c|c|c|c|c|}
\hline
$f(n)$&$g(n)$&$O$&$o$&$\Omega$&$\omega$&$\Theta$\\
\hline
$8^n$&$8^{n/2}$&No&No&Yes&Yes&No\\
\hline
$\sqrt{n^k} $(k being an even integer)$ $&$2^n$&Yes& Yes&No&Yes&No\\
\hline
$\log(n^n)$&$ln(n!)$\\
\hline
\end{tabular}\\
$$\lim_{n\to\infty}\frac{f(n)}{g(n)}=
\begin{cases} 
      C \neq 0 & f(n) \in \Theta(g(n)) \leftrightarrow g(n) \in \Theta(f(n))\\
      0 & f(n) \in o(g(n)) \rightarrow f(n) \in O(g(n))\\
      \infty & f(n) \in \omega(g(n)) \rightarrow f(n) \in \Omega(g(n))\\
   \end{cases}
\,$$\\
$f(n) \in \omega(g(n)) \leftrightarrow f(n) \in o(g(n))$\\
$\lim_{x\to\infty}\frac{8^x}{8^{x/2}}\rightarrow\lim_{x\to\infty}8^{x/2}=\infty\therefore 8^n \in \Omega (8^{x/2}), \in \omega(8^{n/2})$\\
$\lim_{x\to\infty} \frac{\sqrt{x^k}}{2^x}$\\
$\sqrt{1x^k}$ results in a polynomial function and $2^n$ is exponential\\
$\therefore \lim_{x\to\infty} \frac{\sqrt{x^k}}{2^x} = 0$ and is $o(g(n)), \omega (g(n))$ and $O(g(n))$\\
\question
From the following functions \\$ \sqrt{2}^{\log n},n^2,n!,\frac{3}{2}^n,n^3,\log^2 n, \log n!, \ln(\ln(n)), n*2^n, \ln n, 1,2^{\log(n)}, e^n, 4^{\log n},$ \\ $ (n+1)!, \sqrt{\log n}, n, 2^n,  n \log n$\\
\begin{itemize}
	\item[1] 
	Rank exponential functions according to their bases\\
	$  \sqrt{2}^{\log n}, \frac{3}{2}^n,  n*2^n,2^{\log(n)}, 4^{\log n}, 2^n,$
	\item[2]
	Partition the functions into functions where $ f(n)\in \Theta g(n) $\\
\end{itemize}
\question
$T ( n ) = 2 T ( n-1 ) + 1 ; T ( 1 ) = 5 .$\\
$T(n) = 2T(n-1)+1 =  2(2T(n-2))+1 = 2^2T(n-2)+2+1$\\
$=2^3T(T(n-3)+2^2+2+1 = 2^4T(n-4)+2^3+2^2+2+1$\\
$=...=2^{n-1}T(1)+2^{n-2}+2^{n-3}+...+2+1$\\
$=2^{n-1}*5+(1+2+2^2+...+2^{n-2})=2^{n-1}*5+\frac{2^{n-1}-1}{1}$\\
$=6*2^{n-1}-1$
\question
For the recurrence $T(n)=
\ \begin{cases} 
      1 & n =1\\
      3T(\frac{n}{3} +1 & n >1\\
   \end{cases}
\,$ show that $T(n) \in \Theta(n)$
where $=3^k$\\
\newpage
\question
Use the recursion tree method to find a big-$ \Theta $ estimate for the solution of the recurrence$ T(n)=3T(\frac{n}{3})+n $\\
\includegraphics[scale=.1, angle=270]{homework2_number7}
\question
Use the master method to give tight asymptotic bounds for the following recurrences\\
$ a < b^d \rightarrow T(n) \in  \Theta(n^d)$\\
$ a=b^d  \rightarrow T(n) \in \Theta (n^d\log n)$\\
$ a>b^d \rightarrow T(n) \in \Theta(n^{\log_b a}) $\\
\begin{alphaparts}
	\item 
	$ T(n) = 2T(\frac{n}{4})+1 $\\
	$ a = 2, b = 4, d=0 $\\
	$ 2 > 4^0 \rightarrow T(n) \in \Theta(n^{\log_b a})$\\
	\item 
	$ T(n)=2T(n/4) + \sqrt{n} $\\
	$ a=2,b=4,d=\frac{1}{2} $\\
	$ 2=4^{\frac{1}{2}} \rightarrow T(n) \in \Theta(n^d\log n) $\\
	\item 
	$ T(n) = 2T(\frac{n}{4})+n $\\
	$ a = 2, b = 4, d=1 $\\
	$ 2 < 4^1 \rightarrow T(n) \in \Theta(n^d)$\\
	\item 
	$ T(n) = 2T(\frac{n}{4})+n^2 $\\
	$ a = 2, b = 4, d=2 $\\
	$ 2 < 4^2 \rightarrow T(n) \in \Theta(n^d)$\\
\end{alphaparts}
\question
\begin{alphaparts}
	\item For the given version of a QUICKSORT algorithm, are arrays made up of all equal elements the worst-case input, the best-case input or neither?\\
	Worst-case, the pivot of the array will always be at element n of the larger sub, meaning one sub array being size 1 and the other being n-1, stepping down through all of the elements until the largest sub array is size 1
	\item
	For the given version of a QUICKSORT algorithm, are strictly decreasing arrays the worst-case input, the best-case input or neither?\\
	Will behave inversely of the quick sort of an array of all equal elements but will have the same run time.\\
	Worst-case, the pivot of the array will always be at element 1 of the larger sub array, meaning one sub array being size 1 and the other being n-1, stepping up through all of the elements until the largest sub array is size 1.\\
\end{alphaparts}
\end{document}