\documentclass[12pt,largemargins]{homework}

  \newcommand{\hwname}{Phillip Janowski}
  \newcommand{\hwemail}{pajmc2@mail.umsl.edu}
  \newcommand{\hwtype}{Homework}
  \newcommand{\hwnum}{1}
  \newcommand{\hwclass}{CS3130}

\usepackage{lipsum}
\usepackage{lineno}
\usepackage{amsmath}

\date{Septemeber 6, 2018}
\begin{document}
\maketitle


\question
  \begin{alphaparts}

  \item
    Find $gcd(213486, 5423)$

    \begin{linenumbers}
      213486 / 5243 = 39 R 1989 \\
      213486 = 5423 * 39 + 1989 \\
      213486 / 1989 = 107 R 663 \\
      213486 = 1989 * 107 + 663 \\
      213486 / 663 = 332 R 0 

    \end{linenumbers}

    $\therefore gcd(213486,5423) = 663 $


  \item
    Estimate approximately how many times faster it will be to find 
    $gcd(213486, 5423)$ with the help of the Euclid’s algorithm compared with the algorithm based on checking consecutive integers from min{m, n} down to gcd(m, n) (see the algorithm \#2 from the handout). You may only count the number of modulus divisions of the largest integer by different divisors.


    \begin{linenumbers}
      $min(213486,5423) = 5423$ \\
      $m/t \neq 0 $ where $m=213486$ and $ t=5423 $
      $t-1 = 5422 $ \\
      $m-gcd(213486, 5423) = 4760$ steps \\
      $4760$ steps $/4$ steps $=1190$ times faster 
    \end{linenumbers}

\end{alphaparts}
\end{document}