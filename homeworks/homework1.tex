\documentclass[12pt,largemargins]{homework}

  \newcommand{\hwname}{Phillip Janowski}
  \newcommand{\hwemail}{pajmc2@mail.umsl.edu}
  \newcommand{\hwtype}{Homework}
  \newcommand{\hwnum}{1}
  \newcommand{\hwlecture}{E}
  \newcommand{\hwsection}{01}
  \newcommand{\hwclass}{CS3130}

\usepackage{lipsum}
\usepackage{lineno}
\usepackage{amsmath}
\usepackage{listings}

\date{Septemeber 6, 2018}
\begin{document}
\maketitle
\normalsize

\question
  \begin{alphaparts}

  \item
    Find $gcd(213486, 5423)$

      $213486 / 5243 = 39 R 1989$ \\*
      $213486 = 5423 * 39 + 1989$ \\*
      $213486 / 1989 = 107 R 663$ \\*
      $213486 = 1989 * 107 + 663$ \\*
      $213486 / 663 = 332 R 0 $

    $\therefore gcd(213486,5423) = 663 $ 


  \item
    Estimate approximately how many times faster it will be to find
    $gcd(213486, 5423)$ with the help of the Euclid’s algorithm compared with the algorithm based on checking consecutive integers from min{m, n} down to gcd(m, n) (see the algorithm \#2 from the handout). You may only count the number of modulus divisions of the largest integer by different divisors.


      $min(213486,5423) = 5423$ \\
      $m/t \neq 0 $ where $m=213486$ and $ t=5423 $\\
      $t-1 = 5422 $ \\
      $m-gcd(213486, 5423) = 4760$ steps \\
      $4760$ steps $/4$ steps $=1190$ times faster 

\end{alphaparts}

  \question
    \begin{arabicparts}

      \item
        Prove the formula on which Euclid's algorithm is based:
        $gcd(m,n)=gcd(n,m$ mod $ n)$ for every pair of positive integers $m$ and $n$\\
        Solution: \\
        We know $gcd(m,0)=m$ and $gcd(0,n)=n$\\
        if $m = n * I + R$ where I is some integer and $N \neq 0$\\
        then $gcd(m,n) = gcd(n,R)$ where remainder $R=m$ mod $n$ \\
        $\therefore gcd(m,n)=gcd(n,m$ mod $n)$
		
\newpage
    
      \item
        2.1-4\\ Given two array's $A$ and $B$, length $n$, are added together and stored in an $(n+1)$ length array, $C$
        \begin{verbatim}
			carried = 0 
			for i = n-1 down to 0 
			     C[i+1] = {A[i] + B[i] + carried} mod 2
			     carried = {A[i] + B[i] + carried} / 2 
			C[0] = carried
        \end{verbatim}
       		

		\item
		2.2-2
		\begin{alphaparts}
		\item
		\begin{verbatim}
		for i = 0 to n - 1
			     smallest = 0
			     smallestElement = 0
			     for 0 = i to n -1
			          if A[j]< A[j+1]
			               smallest = A[j]
			               smallestElement = j
			     A[smallestElement] = j
			     A[i] = smallest
		\end{verbatim}
		\item
		The loop is invariant because at the beginning of each for loop, the 
		sorted part of the array changes in size with each iteration and only have the i-1 smallest elements 
		\item
		The loop only needs to run for n-1 times because the last element will already be sorted
		\item
		The run time of this loop is $\theta (n^2)$, because the inner loop runs each of its lines $n$ times and the out loop runs its lines, including the inner loop, $n$ times. $\therefore \theta(n^2)$

		\end{alphaparts}
\end{arabicparts}

\newpage
\question
Modify the psuedo code on p.40 to count swaps for the bubble sort
\begin{verbatim}
i = 0
swap = 1
while swap != 0 and i < A.length
     swap = 0
          for j = A.length down to i + 1
               if A[j] < A[j-1]
                    exchange A[j] < A[j-1]
                    swap++
                    i++ 
\end{verbatim}

\question
	Handwritten and Stapled to the packet
	\begin{alphaparts}
	\item
	Insertion Sort
	\begin{center}
		\begin{tabular}{c c c c c c c c c c c c}
		\hline
		R & E & A & L & I & T & Y & S & H & O & W & Comparisons\\
		\hline
		R & E & A & L & I & T & Y & S & H & O & W & 1\\
		\hline
		E| & R & A & L & I & T & Y & S & H & O & W & 2 \\
		\hline
		A & E| & R & L & I & T & Y & S & H & O & W & 2 \\
		\hline
		A & E & L| & R & I & T & Y& S & H & O & W & 3 \\
		\hline
		A & E & I & L| & R & T & Y & S & H & O & W & 1 \\
		\hline
		A & E & I & L & R| & T & Y & S & H & O & W & 1 \\
		\hline
		A & E & I & L & R & T| & Y & S & H & O & W & 3 \\
		\hline
		A & E & I & L & R & S & T| & Y & H & O & W & 7 \\
		\hline
		A & E & H & I & L & R & S & T| &  Y  & O & W & 5 \\
		\hline
		A & E & H & I & L & O & R & S & T| &  Y  & W & 2 \\
		\hline
		A & E & H & I & L & O & R & S & T &  W|  & Y & 27 total \\
		\end{tabular}
	\end{center}
	\newpage
	\item
	Selection Sort
	\begin{center}
		\begin{tabular}{c c c c c c c c c c c c}
		\hline
		R & E & A & L & I & T & Y & S & H & O & W & Comparisons\\
		\hline
		R & E & A & L & I & T & Y & S & H & O & W & 10\\
		\hline
		A| & E & R & L & I & T & Y & S & H & O & W & 9\\
		\hline
		A & E| & R & L & I & T & Y & S & H & O & W & 8\\
		\hline
		A & E & H| & L & I & T & Y & S & R & O & W & 7 \\
		\hline
		A & E & H & I| & L & T & Y & S & R & O & W & 6\\
		\hline
		A & E & H & I & L| & T & Y & S & R & O & W & 5\\
		\hline
		A & E & H & I & L & O| & Y & S & R & T & W & 4\\
		\hline
		A & E & H & I & L & O & R| & S & Y & T & W & 3\\
		\hline
		A & E & H & I & L & O & R & S| & Y & T & W & 2\\
		\hline
		A & E & H & I & L & O & R & S & T| & Y & W & 1\\
		\hline
		A & E & H & I & L & O & R & S & T & W| & Y & 55 total\\
		\end{tabular}
	\end{center}
	\item
	Bubble Sort With Swap Count
		\begin{center}
		\begin{tabular}{c c c c c c c c c c c c c }
		R & E & A & L & I & T & Y & S & H & O & W & Comparisons & Swaps\\
		\hline
		R & E & A & L & I & T & Y & S & H & O & W & 10 & 8\\
		\hline
		E & A & L & I & R & T & S & H & O & W & |Y & 9 &4\\
		\hline
		A & E & I & L & R & S & H & O & T & |W & Y & 8 & 2\\
		\hline
		A & E  & I & L & R & H & O & S & |T & W & Y & 7 & 2\\
		\hline
		A & E & I & L & H & O & R & |S & T & W & Y & 6 & 1\\
		\hline
		A & E & I & H & L & O & |R & S & T & W & Y & 5 & 1\\
		\hline
		A & E & H & I & L & |O & R & S & T & W & Y & 4 & 0\\
		\hline
		A & E & H & I & L & O & R & S & T & W & Y & 49 total & 14 total \\
		\end{tabular}
	\end{center}
	\newpage

		\item
	Bubble Sort Without Swap Count
		\begin{center}
		\begin{tabular}{c c c c c c c c c c c c c }
		\hline
		R & E & A & L & I & T & Y & S & H & O & W & Comparisons & Swaps\\
		\hline
		R & E & A & L & I & T & Y & S & H & O & W & 10 & 8\\
		\hline
		E & A & L & I & R & T & S & H & O & W & |Y & 9 &4\\
		\hline
		A & E & I & L & R & S & H & O & T & |W & Y & 8 & 2\\
		\hline
		A & E  & I & L & R & H & O & S & |T & W & Y & 7 & 2\\
		\hline
		A & E & I & L & H & O & R & |S & T & W & Y & 6 & 1\\
		\hline
		A & E & I & H & L & O & |R & S & T & W & Y & 5 & 1\\
		\hline
		A & E & H & I & L & |O & R & S & T & W & Y & 4 & 0\\
		\hline
		A & E & H & I & |L & O & R & S & T & W & Y & 3 & 0\\
		\hline
		A & E & H & |I & L & O & R & S & T & W & Y & 2 & 0\\
		\hline
		A & E & |H & I & L & O & R & S & T & W & Y & 1 & 0\\
		\hline
		A & E & H & I & L & O & R & S & T & W & Y & 55 total & 14 total \\
		\end{tabular}
	\end{center}
	\end{alphaparts}
	
\question
	\begin{alphaparts}
	\item
	\begin{verbatim}
		i = 0
		while i < 3n
		     print"CS3130"
		          i = i + 3
	\end{verbatim}
		$=\sum_{i=0}^{(3n-3)/3} 1$
		$=(n-1+0-1) + 3(3n-0+1)$ \large{$$=n$$}
	\item
		\begin{verbatim}
		for i = 0 to n
		     for j = 0 to n 
		          print"CS3130"
		\end{verbatim}
		$=\sum_{i=0}^{n-1} \sum_{j=0}^{n-1} 1$
		$=\sum_{i=0}^{n-1} (n-1-0+1)$
		$=n\sum_{i=0}^{n-1} 1 $\\
		$=n(n-1-0+1)$
		$$=n^2$$

\newpage
	\item
	\begin{verbatim}
	for i = 0 to n 
	     for j = i to n
	          print"CS3130"
	\end{verbatim}
	$=\sum_{i=0}^{n-1} \sum_{j=i}^{n-1} 1$
	$=\sum_{i=0}^{n-1} (n-1-i+1)$
	$=n\sum_{i=0}^{n-1} 1 - \sum_{i=0}^{n-1} i$\\
	$=n(n-1-0+1) - \dfrac{(0 + n-1)}{2} (n -1- 0+1)$
	$=n^2 - \dfrac{n^2 - n}{2}$ \\
	$$=\dfrac{n^2 + n}{2}$$

	\end{alphaparts}
	
\question
2.1-3
Find v in Array A[ ]

\begin{verbatim}
i = 0
while i < A.length and v != A[i]
     i++
if A[i] != v
     i = null
\end{verbatim}
This loop is invariant because at the beginning we have $i = 0$ at ininitialization. Maintenance, because it maintains while $i < length$ of A or $v \neq A[i]$ and is true before i and i + 1. And Termination, because it only terminates when v is found in A or when we run out of elements in A to check
\question
The average performance of the linear search is $O(\dfrac{n}{2})$. The worst performance would happen from v being in the $A.length$th position of the array and the best would happen from v being in the 1st position of the array. In a set of completely random runs the average number of comparisons would eventually converge on $\dfrac{n}{2}$.
\end{document}