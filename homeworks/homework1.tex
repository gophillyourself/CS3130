\documentclass[12pt,largemargins]{homework}

  \newcommand{\hwname}{Phillip Janowski}
  \newcommand{\hwemail}{pajmc2@mail.umsl.edu}
  \newcommand{\hwtype}{Homework}
  \newcommand{\hwnum}{1}
  \newcommand{\hwlecture}{E}
  \newcommand{\hwsection}{01}
  \newcommand{\hwclass}{CS3130}

\usepackage{lipsum}
\usepackage{lineno}
\usepackage{amsmath}
\usepackage{listings}

\date{Septemeber 6, 2018}
\begin{document}
\maketitle


\question
  \begin{alphaparts}

  \item
    Find $gcd(213486, 5423)$

    \begin{linenumbers}
      $213486 / 5243 = 39 R 1989$ \\*
      $213486 = 5423 * 39 + 1989$ \\*
      $213486 / 1989 = 107 R 663$ \\*
      $213486 = 1989 * 107 + 663$ \\*
      $213486 / 663 = 332 R 0 $

    \end{linenumbers}

    $\therefore gcd(213486,5423) = 663 $ 


  \item
    Estimate approximately how many times faster it will be to find
    $gcd(213486, 5423)$ with the help of the Euclid’s algorithm compared with the algorithm based on checking consecutive integers from min{m, n} down to gcd(m, n) (see the algorithm \#2 from the handout). You may only count the number of modulus divisions of the largest integer by different divisors.


    \begin{linenumbers}
      $min(213486,5423) = 5423$ \\
      $m/t \neq 0 $ where $m=213486$ and $ t=5423 $
      $t-1 = 5422 $ \\
      $m-gcd(213486, 5423) = 4760$ steps \\
      $4760$ steps $/4$ steps $=1190$ times faster 
    \end{linenumbers}

\end{alphaparts}

  \question
    \begin{arabicparts}

      \item
        Prove the formula on which Euclid's algorithm is based:
        $gcd(m,n)=gcd(n,m$ mod $ n)$ for every pair of positive integers $m$ and $n$

\newpage
    
      \item
        2.1-4\\ Given two array's $A$ and $B$, length $n$, are added together and stored in an $(n+1)$ length array, $C$
        \begin{verbatim}
			carried = 0 
			for i = n-1 down to 0 
			     C[i+1] = {A[i] + B[i] + carried} mod 2
			     carried = {A[i] + B[i] + carried} / 2 
			C[0] = carried
        \end{verbatim}
       		

		\item
		2.2-2
		\begin{alphaparts}
		\item
		\begin{verbatim}
		for i = 0 to n - 1
			     smallest = 0
			     smallestElement = 0
			     for 0 = i to n -1
			          if A[j]< A[j+1]
			               smallest = A[j]
			               smallestElement = j
			     A[smallestElement] = j
			     A[i] = smallest
		\end{verbatim}
		\item
		The loop is invariant because at the beginning of each for loop, the 
		sorted part of the array changes in size with each iteration and only have the i-1 smallest elements 
		\item
		The loop only needs to run for n-1 times because the last element will already be sorted
		\item
		The run time of this loop is $\theta (n^2)$, because the inner loop runs each of its lines $n$ times and the out loop runs its lines, including the inner loop, $n$ times. $\therefore \theta(n^2)$

		\end{alphaparts}
\end{arabicparts}

\newpage
\question
Modify the psuedo code on p.40 to count swaps for the bubble sort
\begin{verbatim}
i = 0
swap = 1
while swap != 0 and i < A.length
     swap = 0
          for j = A.length down to i + 1
               if A[j] < A[j-1]
                    exchange A[j] < A[j-1]
                    swap++
                    i++ 
\end{verbatim}

\question
	Handwritten and Stapled to the packet
\question
	\begin{alphaparts}
	\item
	\begin{verbatim}
		i = 0
		     while i < 3n
		          print"CS3130"
		          i = i + 3
	\end{verbatim}
		The answer is n times though I came up with 9n+3
	\item
		\begin{verbatim}
		for i = 0 to n
		     for j = 0 to n 
		          print"CS3130"
		\end{verbatim}
		$=\sum_{i=0}^{n} \sum_{j=0}^{n} 1$
		$=\sum_{i=0}^{n} (n-0+1-1)$
		$=n\sum_{i=0}^{n} 1 $
		$=n(n-0+1-1)$
		$$=n^2$$

\newpage
	\item
	\begin{verbatim}
	for i = 0 to n 
	     for j = i to n
	          print"CS3130"
	\end{verbatim}
	$=\sum_{i=0}^{n} \sum_{j=i}^{n} 1$
	$=\sum_{i=0}^{n} (n-i+1-1)$
	$=n\sum_{i=0}^{n} 1 - \sum_{i=0}^{n} i$\\
	$=n(n-0+1) - \dfrac{(0 + n)}{2} (n - 0 + 1)$
	$=n^2 + n - \dfrac{n^2 + n}{2}$ \\
	$$=\dfrac{n^2 + n}{2}$$

	\end{alphaparts}
	
\question
2.1-3
Find v in Array A[ ]

\begin{verbatim}
i = 0
while i < A.length and v != A[i]
     i++
if A[i] != v
     i = null
\end{verbatim}
This loop is invariant because at the beginning we have $i = 0$ at ininitialization. Maintenance, because it maintains while $i < length$ of A or $v \neq A[i]$ and is true before i and i + 1. And Termination, because it only terminates when v is found in A or when we run out of elements in A to check
\question
The average performance of the linear search is $O(\dfrac{n}{2})$. The worst performance would happen from v being in the $A.length$th position of the array and the best would happen from v being in the 1st position of the array. In a set of completely random runs the average number of comparisons would eventually converge on $\dfrac{n}{2}$.
\end{document}